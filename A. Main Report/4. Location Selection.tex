\section{Location Selection}
\vspace{-10pt}
\subsection{Objective}
\subsection{Location Options}
\subsection{Results and Decisions}
\subsection{Sensitivity}
\subsection{Key Assumptions}



MCDM analysis was also utilised to determine key decisions within the process beyond the synthesis route. Key decisions included the potential reactor units for each reaction, waste management, and whether to outsource the brine or not. Similar criteria were used to make these decisions to those described in in Table \ref{tab:TOPSIS1}. Figure \ref{fig:GenOverview} gives an overview of the process being developed by Solution A.

\begin{figure}[H]
    \centering
    \includegraphics[width=170mm]{Figures/Process_Overview.png}
    \caption{Overview of the Solution A modified Solvay process.}
    \label{fig:GenOverview}
    
\end{figure}

\vspace{-15pt}
\subsection{Limestone Calcination}
\vspace{-10pt}
\label{Limestone Calcination}
The limestone fed to the process is calcined at 1000 \textdegree C using a DSR. This produces CaO, which is divided between the calcium loop and the primary solvay process, as well as a stream of pure CO$_2$, which is split between third party carbon sequestration and the primary solvay process. Natural gas is used as the fuel burned in the DSR to provide the necessary heat, and this produces  a flue gas that is sent to the carbon capture loop. 
At this stage, the limestone is assumed to be pure CaCO$_3$, although in reality impurities will be present and will be addressed in subsequent design.
\vspace{-15pt}
\subsection{Carbon Capture}
\vspace{-10pt}
\label{Carbon Capture}
A calcium loop is used to remove CO$_2$ from the DSR flue gas, hence greatly reducing emissions. Hot flue gas from the two DSRs is reacted with a portion of the CaO from the limestone calcination, producing CaCO$_3$ and leaving the remaining air clean. This air is vented whilst the CaCO$_3$ is recycled back into the limestone calciner.
\vspace{-10pt}
\subsection {Sodium Bicarbonate Production} 
\vspace{-10pt}
The remaining CaO is first sent to a slaker, where it is reacted with water to produce a basic solution of Ca(OH)$_2$. This solution is mixed with the brine feed, which contains approximately 30 wt\% NaCl, and fed to a bubble column. In this column, the CO$_2$ from the limestone calcination is bubbled through the solution, forming CaCl$_2$ and precipitating NaHCO$_3$. This slurry is then filtered, yielding a NaHCO$_3$ filter cake and CaCl$_2$-rich byproduct solution.
\vspace{-10pt}
\subsection {Sodium Carbonate Production} 
\vspace{-10pt}
\label{Sodium Bicarbonate Production}
Finally, the Na$_2$CO$_3$ (soda ash) product is produced by calcining the NaHCO$_3$ in a second DSR. The product produced in this reaction was found to be a sufficient purity for sale, and so further processing was not needed. The calcination also produces steam and CO$_2$, these are separated and the CO$_2$ recycled back to the bubble column. Once again, fuel is burned in order to provide the heat requirement for this DSR, and the flue gas is also sent to the carbon capture loop.
\vspace{-15pt}