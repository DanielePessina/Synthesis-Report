\section{Introduction}
\vspace{-10pt}
Sodium carbonate, also known as soda ash, is an important salt that takes the form of a white powder at standard conditions, with the formula Na$_2$CO$_3$. It finds a range of uses, including in detergents, production of other inorganic chemicals, and in particular as flux in the production of soda-lime glass where it decreases the melt temperature of the silica \citep{SodiumCarb1}\citep{Thieme2000}. Since antiquity, sodium carbonate was extracted from the ashes of sodium-rich plants, hence the name ‘soda ash’ \citep{Thieme2000}. However, with the industrialisation of Europe came an ever-increasing demand for the salt, which could not be met using traditional methods. The LeBlanc process was the first scalable process developed for the production of sodium carbonate, with the reaction steps shown in Figure \ref{fig:LeblancProcess}. This process was very environmentally damaging due to the emission of HCl gas and CaS \citep{Thieme2000}, and the low atom economy 31.3\% is indicative of the economic downsides of this process, as the by-products were at the time of low value \citep{SodiumCarb1}. 
\vspace{-10pt}
\begin{figure}[H]
    \centering
    \includegraphics[width=100mm]{Figures/Leblanc_Process_Route.png}
    \caption{Leblanc process synthesis pathway.}
    \label{fig:LeblancProcess}
\end{figure}
\vspace{-10pt}
\noindent An alternative process was proposed by Ernest Solvay in the 1860s \citep{Thieme2000}, shown in Figure \ref{fig:SolvayProcess}.
The Solvay process improved significantly upon its predecessor, with an increased atom economy of 48.9\% and no toxic by-products (CaCl$_2$ is considered non-toxic and is commonly released safely into the ocean \citep{Steinhauser2008}). Since the end of the 19th century the majority of the world’s soda ash has been produced via the Solvay process, with the exception of the United States which is able to utilise its natural reserves of the mineral trona, Na$_3$H(CO$_3$)$_2$  .2H$_2$O, from which sodium carbonate can be extracted directly \citep{Wyoming2016}.\\
\begin{figure}[H]
    \centering
    \includegraphics[width=90mm]{Figures/Solvay_Process_Route.png}
    \caption{Solvay process synthesis pathway.}
    \label{fig:SolvayProcess}
\end{figure}
\vspace{-10pt}
\noindent Despite its advantages over the Leblanc process, there remain some downsides with the Solvay process. Firstly, the use of ammonia as a basic intermediate significantly increases the safety and environmental risks associated with the process, as ammonia is known to be highly toxic \citep{ammonia_MSDS}. Secondly, current Solvay processes directly emit between 200 to 400 kg CO$_2$ per ton product, primarily due to the burning of fossil fuels to provide heat to the process \citep{EUDoc2007}. Given the current need to reduce greenhouse gas emissions across all industries, this is a figure which must be reduced over the coming years. 
\\ \\
\noindent Solution A Ltd. seeks to address these key issues in order to deliver a quality product without compromising on our duty to protecting society and the environment. To do this, Solution A Ltd. aims to draw on the lessons learned from the extensive efforts that have sought to decarbonise the cement industry, given that limestone calcination is a key part of both cement manufacture and the Solvay process. In particular, the development of the direct separation reactor (DSR) for calcination by Calix \citep{LEILAC} proves promising as a low-complexity modification to the traditional process, capable of reducing emissions by up to 60\% \cite{Hills2015a}.\\ \\
\noindent This report outlines the initial design of a Solvay process plant, with a target production rate of 1 Mtpa high-purity (>99 wt\%) sodium carbonate and CO2 emission rate of at most 30 kg per ton product. Potential options and key decisions are outlined, culminating in the final process flow sheet which is then used for proof-of-concept and economic analysis.
\vspace{-10pt}

