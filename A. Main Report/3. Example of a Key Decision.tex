\section{Example of a Key Decision}

\subsection{Objective}

\subsection{Alternatives}

\subsection{MCDM Results }

\vspace{-10pt}
To ensure the synthesised product meets all of our client’s specifications in the most effective way, Solution A devised and compared a multitude of potential process routes, both qualitatively and quantitatively. Where applicable, multi-criteria decision making (MCDM) was used to determine the optimal decision from a set of potential options. In MCDM, an appropriate set of criteria are identified and quantified, based on information from literature and pairwise comparisons. Both the AHP and TOPSIS methods were used. In other cases, decisions were made using established heuristics and informed by literature on existing Solvay and similar processes. The following sections, alongside Appendix \ref{appendixtopsis}, gives detailed descriptions on Solution A\textquotesingle s proposed synthesis route for soda ash. Product specifications were assigned to hit a target purity of 99.0\% and production rate of at least 100 Mt per annum of soda ash.

\vspace{-10pt}
\subsection{MCDM Comparison Criteria}
\vspace{-10pt}
For the MCDM analysis of process routes seven criteria were chosen to compare potential options, summarised in Table \ref{tab:TOPSIS1}. Weightings for these criteria were determined through AHP, with the same weightings applied to both AHP and TOPSIS analysis.
\vspace{-10pt}
\begin{table}{H}
\centering\small
\caption{Criteria and weightings associated with selecting the appropriate synthesis route}
\begin{tabular}{|l|l|c|}
\hline
\textbf{Criteria}                                                                & \textbf{Description}                                                                                                                                                                                & \multicolumn{1}{l|}{\textbf{Weighting}} \\ \hline
Novelty                                                                          & \begin{tabular}[c]{@{}l@{}}A measure of how innovative the process is compared to \\ existing Solvay processes.\end{tabular}                                                                        & 0.11                                    \\ \hline
\multirow{CO$_2$ Capture Rate}                                             & \begin{tabular}[c]{@{}l@{}}(Solvay Route) Amount of CO$_2$ uptake in the main \\ Solvay reactor.\end{tabular}                                                                                       & \multirow{0.22}                   \\ \cline{2-2}
                                                                                 & \begin{tabular}[c]{@{}l@{}}(Carbon Abatement) \% of CO$_2$ captured from \\ process flue gas.\end{tabular}                                                                                          &                                         \\ \hline
\begin{tabular}[c]{@{}l@{}}Technology \\ Readiness\\ Level\end{tabular}             & \begin{tabular}[c]{@{}l@{}}A score of 1-9 rating the readiness for an option to be \\ deployed at scale, with 1 being the initial research stage \\ and 9 being full-scale deployment.\end{tabular} & 0.14                                    \\ \hline
\multirow{Cost}                                                            & (Solvay Route) Cost of reagents.                                                                                                                                                                    & \multirow{0.16}                   \\ \cline{2-2}
                                                                                 & \begin{tabular}[c]{@{}l@{}}(Carbon Abatement) Cost of mitigation technology, \\ based on a 1 Mtpa cement plant.\end{tabular}                                                                        &                                         \\ \hline
\multirow{\begin{tabular}[c]{@{}l@{}}Safety\\ Considerations\end{tabular}} & \begin{tabular}[c]{@{}l@{}}(Solvay Route) Toxicity of reagents used, specifically \\ to aquatic life taken from MSDS sheets as a basis for \\ general toxicity.\end{tabular}                        & \multirow{0.24}                   \\ \cline{2-2}
                                                                                 & \begin{tabular}[c]{@{}l@{}}(Carbon Abatement) Score of 1-9 accounting for \\ safety risks of any new chemicals/process units needed \\ for the technology,\end{tabular}                             &                                         \\ \hline
Complexity                                                                       & \begin{tabular}[c]{@{}l@{}}Score of 1-9 accounting for the complexity of \\ process units and recycles required by the technology.\end{tabular}                                                     & 0.08                                    \\ \hline
Footprint                                                                        & \begin{tabular}[c]{@{}l@{}}Score of 1-9 accounting for the physical size of the \\ technology.\end{tabular}                                                                                         & 0.05                                    \\ \hline
\end{tabular}
\label{tab:TOPSIS1}
\end{table}
\vspace{-10pt}
\subsection{Traditional Ammonia-Soda Process}
\vspace{-10pt}
In the 19th century, the Leblanc process dominated the world production of soda ash  \citep{LeBlanc}. Recent interest in green engineering and safety concerns resulted in the complete redundancy of the Leblanc and Trona processes when the Solvay process was commercialised in the 1860s. This helped to reduce and eliminate emissions of waste into the environment such as carbon monoxide (CO), hydrochloric acid (HCl), and calcium sulfide (CaS). Numerous variations of the Solvay process are proposed, including the introduction of the DSR to complement the traditional technology.  
\vspace{-10pt}
\subsubsection{Modified Calcium Oxide Process}
\vspace{-10pt}
Soda ash production via the traditional Solvay process requires the use of ammonia gas to form basic ammoniated brine, which is then passed through to a bubble column where carbon dioxide is introduced. A more novel approach was developed by Solutions A Ltd. whereby ammonia was eliminated all-together (reducing the hazard of NH$_3$ storage) and instead a calcium loop was integrated whereby CaO was dissolved to form basic calcium hydroxide. Appendix \ref{appendixtopsis} shows how the MCDM analysis was performed. 
\vspace{-10pt}
\subsection{Potential Carbon Abatement Routes}
\vspace{-10pt}
It is essential from a business and environmental standpoint that Solution A delivers a low carbon product. To this end, a limit of 30 kg of CO$_2$ per tonne of soda ash has been specified by our client, with an economic penalty of \pounds 250 per tonne CO$_2$ emitted over this limit. Existing soda ash plants currently produce 200-300 kg CO$_2$ per tonne soda ash, and hence Solution A is looking to reduce this by 90\% using CO$_2$ abatement technologies. The core of Solution A’s process is the DSR, which calcines carbonates through indirect heating as opposed to a conventional kiln. It is already estimated that the use of a DSR can reduce process emissions by 60\% , so other technologies were evaluated using MCDM to further mitigate the remaining 40\%. These include amine scrubbing, calcium looping, and oxy-fuel combustion technologies. Unabated kiln and DSR technologies were also considered for comparison.
\vspace{-10pt}
\subsubsection{Unabated Kiln}
\vspace{-10pt}
In traditional lime kilns, fuel is mixed with the limestone and burned, with the heat of combustion directly decomposing the limestone. The result is a flue gas containing the CO$_2$ from both the combustion and decomposition, as well as other combustion products and significant amounts of nitrogen if air is used. The flue gas is too dilute with respect to CO$_2$ for effective sequestration and will introduce impurities into the rest of the process.
\vspace{-10pt}
\subsubsection{Direct Separation Reactor (DSR)}
\vspace{-10pt}
The DSR circumvents the problems associated with an unabated kiln by burning the fuel separately to the limestone. Heat transfer through the reactor wall indirectly decomposes the limestone, resulting in a stream of pure CO$_2$ out of the inner reactor chamber. However, the combustion of fuel in the outer reactor chamber produces a flue gas requiring further abatement.  
\vspace{-10pt}
\subsubsection{Amine Scrubbing}
\vspace{-10pt}
Amine scrubbing involves the absorption of CO$_2$ from a lean CO$_2$ stream and desorption into a separate, pure CO$_2$ stream via the use of an amine-based solvent such as monoethanolamine \citep{AmineScrubbing}. Amine scrubbing is a mature technology with the benefit of being easily appended onto the end of any carbon-emitting process. However, implementation of this technology would greatly increase the complexity of the process with need for both carbon capture and solvent regeneration units. Moreover, a significant inventory of a toxic amine-based solvent would pose significant safety risks, which would undermine the reduction in risk resulting from the use of the CaO pathway over the traditional NH$_3$ pathway.
\vspace{-10pt}
\subsubsection{Calcium Integration}
\vspace{-10pt}
Calcium looping involves the carbonation of CaO to remove CO$_2$ from a lean stream, followed by the decomposition of the resulting carbonate to produce a pure CO$_2$ stream whilst regenerating the oxide. Calcium looping is also a reasonably mature technology, with the major benefit being that it does not introduce any species that were not already present in the Solvay process.
Furthermore, as a limestone calciner is already present in the process, it is proposed that the calcium loop can be integrated into the main process to reduce the sintering of particles from repeated calcination/carbonation. The oxide produced by the DSR would be split between a carbonator and the main Solvay process, with the carbonate from the former being recycled to the DSR, whilst the latter would act as a ‘purge’ stream from the calcium loop.
\vspace{-10pt}
\subsubsection{Oxy-fuel Combustion}
\vspace{-10pt}
The main source of direct carbon emissions is the combustion of natural gas in air to heat the DSR. This produces a flue gas too dilute in CO$_2$ for sequestration or efficient utilization. Oxy-fuel combustion involves separation of air, such that the fuel can be burned in pure oxygen leading to a pure CO$_2$/ H$_2$O flue gas that can easily be separated through a knockout drum. This would allow the CO$_2$ from combustion to be sequestered or used directly, however it would also necessitate the cryogenic separation of air which presents increased cost, complexity, and safety challenges.
\vspace{-10pt}
\subsection{Chosen Abatement Technology}
\vspace{-5pt}
Table \ref{tab:TOPSIS_synth} summarises the results of the MCDM analysis, outlined in detail in Appendix \ref{appendixtopsis}.
\vspace{-10pt}
\begin{table}[H]
\centering
\caption{Final TOPSIS and AHP analysis scores based on chosen criteria and weighting}
\begin{tabular}{|l|cc|cc|}
\hline
\multicolumn{1}{|c|}{\multirow{\textbf{Technology}}} & \multicolumn{2}{c|}{\textbf{TOPSIS}}                & \multicolumn{2}{c|}{\textbf{AHP}}                   \\ \cline{2-5} 
\multicolumn{1}{|c|}{}                                     & \multicolumn{1}{c|}{\textbf{Score}} & \textbf{Rank} & \multicolumn{1}{c|}{\textbf{Score}} & \textbf{Rank} \\ \hline
Unabated kiln                                              & \multicolumn{1}{c|}{0.651}          & 3             & \multicolumn{1}{c|}{0.563}          & 3             \\ \hline
Unabated DSR                                               & \multicolumn{1}{c|}{0.778}          & 2             & \multicolumn{1}{c|}{0.586}          & 2             \\ \hline
DSR with amine scrubbing                                   & \multicolumn{1}{c|}{0.324}          & 5             & \multicolumn{1}{c|}{0.362}          & 5             \\ \hline
\textbf{DSR with calcium integration}                      & \multicolumn{1}{c|}{\textbf{0.793}} & \textbf{1}    & \multicolumn{1}{c|}{\textbf{0.587}} & \textbf{1}    \\ \hline
DSR with oxy-fuel combustion                               & \multicolumn{1}{c|}{0.581}          & 4             & \multicolumn{1}{c|}{0.524}          & 4             \\ \hline
\end{tabular}
\label{tab:TOPSIS_synth}
\end{table}

\noindent Both analyses agree that DSR with Calcium integration is the best choice for carbon abatement. Amine scrubbing and oxy-fuel combustion suffer due to their high cost and safety implications, whereas calcium integration benefits from including similar technologies to those already used in the Solvay process. Note that AHP found the unabated DSR to be a very close second choice to calcium integration, however since the DSR on its own provides insufficient mitigation Solution A is confident in the decision to move forward with calcium integration technology.



 
\vspace{-15pt}



